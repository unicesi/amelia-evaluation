\documentclass[letterpaper,11pt]{article}

\usepackage[a4paper, total={6in, 8in}]{geometry}
\usepackage{authblk} % authors and affiliations
\usepackage{parcolumns} % questions and answers in columns

\usepackage{wasysym}% provides \ocircle and \Box
\usepackage{enumitem}% easy control of topsep and leftmargin for lists
\usepackage{color}% used for background color
\usepackage{forloop}% used for \Qrating and \Qlines
\usepackage{ifthen}% used for \Qitem and \QItem

% ==================================================================
% From: http://www.svenhartenstein.de/creating-questionnaires-with-latex/

%%%%%%%%%%%%%%%%%%%%%%%%%%%%%%%%%%%%%%%%%%%%%%%%%%
%% Beginning of questionnaire command definitions %%
%%%%%%%%%%%%%%%%%%%%%%%%%%%%%%%%%%%%%%%%%%%%%%%%%%
%%
%% 2010, 2012 by Sven Hartenstein
%% mail@svenhartenstein.de
%% http://www.svenhartenstein.de
%%
%% Please be warned that this is NOT a full-featured framework for
%% creating (all sorts of) questionnaires. Rather, it is a small
%% collection of LaTeX commands that I found useful when creating a
%% questionnaire. Feel free to copy and adjust any parts you like.
%% Most probably, you will want to change the commands, so that they
%% fit your taste.
%%
%% Also note that I am not a LaTeX expert! Things can very likely be
%% done much more elegant than I was able to. If you have suggestions
%% about what can be improved please send me an email. I intend to
%% add good tipps to my website and to name contributers of course.
%%
%% 10/2012: Thanks to karathan for the suggestion to put \noindent
%% before \rule!

%% \Qq = Questionaire question. Oh, this is just too simple. It helps
%% making it easy to globally change the appearance of questions.
\newcommand{\Qq}[1]{\textbf{#1}}

%% \QO = Circle or box to be ticked. Used both by direct call and by
%% \Qrating and \Qlist.
\newcommand{\QO}{$\Box$}% or: $\ocircle$

%% \Qrating = Automatically create a rating scale with NUM steps, like
%% this: 0--0--0--0--0.
\newcounter{qr}
\newcommand{\Qrating}[1]{\QO\forloop{qr}{1}{\value{qr} < #1}{---\QO}}

%% \Qline = Again, this is very simple. It helps setting the line
%% thickness globally. Used both by direct call and by \Qlines.
\newcommand{\Qline}[1]{\noindent\rule{#1}{0.6pt}}

%% \Qlines = Insert NUM lines with width=\linewith. You can change the
%% \vskip value to adjust the spacing.
\newcounter{ql}
\newcommand{\Qlines}[1]{\forloop{ql}{0}{\value{ql}<#1}{\vskip0em\Qline{\linewidth}}}

%% \Qlist = This is an environment very similar to itemize but with
%% \QO in front of each list item. Useful for classical multiple
%% choice. Change leftmargin and topsep accourding to your taste.
\newenvironment{Qlist}{%
\renewcommand{\labelitemi}{\QO}
\begin{itemize}[leftmargin=1.5em,topsep=-.5em]
}{%
\end{itemize}
}

%% \Qtab = A "tabulator simulation". The first argument is the
%% distance from the left margin. The second argument is content which
%% is indented within the current row.
\newlength{\qt}
\newcommand{\Qtab}[2]{
\setlength{\qt}{\linewidth}
\addtolength{\qt}{-#1}
\hfill\parbox[t]{\qt}{\raggedright #2}
}

%% \Qitem = Item with automatic numbering. The first optional argument
%% can be used to create sub-items like 2a, 2b, 2c, ... The item
%% number is increased if the first argument is omitted or equals 'a'.
%% You will have to adjust this if you prefer a different numbering
%% scheme. Adjust topsep and leftmargin as needed.
\newcounter{itemnummer}
\newcommand{\Qitem}[2][]{% #1 optional, #2 notwendig
\ifthenelse{\equal{#1}{}}{\stepcounter{itemnummer}}{}
\ifthenelse{\equal{#1}{a}}{\stepcounter{itemnummer}}{}
\begin{enumerate}[topsep=2pt,leftmargin=0em]
\item[\textbf{\arabic{itemnummer}#1.}] #2
\end{enumerate}
}

%% \QItem = Like \Qitem but with alternating background color. This
%% might be error prone as I hard-coded some lengths (-5.25pt and
%% -3pt)! I do not yet understand why I need them.
\definecolor{bgodd}{rgb}{0.8,0.8,0.8}
\definecolor{bgeven}{rgb}{0.9,0.9,0.9}
\newcounter{itemoddeven}
\newlength{\gb}
\newcommand{\QItem}[2][]{% #1 optional, #2 notwendig
\setlength{\gb}{\linewidth}
\addtolength{\gb}{-5.25pt}
\ifthenelse{\equal{\value{itemoddeven}}{0}}{%
\noindent\colorbox{bgeven}{\hskip-3pt\begin{minipage}{\gb}\Qitem[#1]{#2}\end{minipage}}%
\stepcounter{itemoddeven}%
}{%
\noindent\colorbox{bgodd}{\hskip-3pt\begin{minipage}{\gb}\Qitem[#1]{#2}\end{minipage}}%
\setcounter{itemoddeven}{0}%
}
}

%%%%%%%%%%%%%%%%%%%%%%%%%%%%%%%%%%%%%%%%%%%%%%%%%%
%% End of questionnaire command definitions %%
%%%%%%%%%%%%%%%%%%%%%%%%%%%%%%%%%%%%%%%%%%%%%%%%%%
\newcommand\amelia[0]{\textsc{Amelia}}
\newcommand\ie[0]{\textit{i.e.}}
\newcommand{\question}[1]{\colchunk{\begin{description}[style=unboxed,leftmargin=0cm]\item{#1}\end{description}}}
\newcommand{\answer}[1]{\colchunk{\begin{description}[style=unboxed,leftmargin=0cm]\item{#1}\end{description}}\colplacechunks}

\title{\textbf{Amelia Evaluation}\\User Questionnaire}

\author[1]{Miguel Jim\'{e}nez}
\author[2]{Luis F. Rivera}
\author[2]{Norha M. Villegas}
\author[2]{Gabriel Tamura}
\author[1]{Hausi M\"{u}ller}
\affil[1]{University of Victoria, Victoria, Canada}
\affil[2]{Universidad Icesi, Cali, Colombia}
\date{}
\setcounter{Maxaffil}{0}
\renewcommand\Affilfont{\itshape\small}

\begin{document}
\maketitle

\Qitem{For each exercise, please indicate the time you spent (in minutes) specifying it. \newline E1. \Qline{1cm} E2. \Qline{1cm} E3. \Qline{1cm} E4. \Qline{1cm}}

\Qitem{Please describe the difficulties you experienced developing this workshop. \Qlines{3}}

\Qitem{How much time have you been working as a professional Software Engineer? \Qline{2.5cm}}

\Qitem{Please describe how much experience you have deploying software. \Qlines{3}}

\Qitem{Please describe how much experience you have with \amelia{}. \Qlines{3}}

\noindent
Please indicate your agreement or disagreement with the following statements by selecting only one square. The left-most square indicates that you \textbf{strongly agree}, while the right-most square indicates that you \textbf{strongly disagree}.

\paragraph{Functional Suitability}
\begin{parcolumns}[colwidths={1=0.76\textwidth},nofirstindent]{2}
	\question{\Qitem{All concepts and building-blocks for solving problems in the software deployment domain can be expressed in \amelia{}.}}
	\answer{\Qrating{5}}
	\colplacechunks
	
	\question{\Qitem{\amelia{} is an appropriate and useful tool for deploying software.}}
	\answer{\Qrating{5}}
	\colplacechunks
\end{parcolumns}

\paragraph{Usability}
\begin{parcolumns}[colwidths={1=0.76\textwidth},nofirstindent]{2}
	\question{\Qitem{The language elements are understandable (\textit{e.g.}, language elements can be understood after reading their descriptions).}}
	\answer{\Qrating{5}}
	\colplacechunks
	
	\question{\Qitem{The concepts and symbols of the language resemble the terminology of the deployment domain, are learnable and rememberable (\ie{}, learning easiness, easiness for developing deployment specifications).}}
	\answer{\Qrating{5}}
	\colplacechunks
	
	\question{\Qitem{\amelia{} helps users achieve their tasks in acceptable development times.}}
	\answer{\Qrating{5}}
	\colplacechunks
	
	\question{\Qitem{\amelia{} is appropriate for the deployment of the type of software you work on.}}
	\answer{\Qrating{5}}
	\colplacechunks
	
	\question{\Qitem{\amelia{} has useful language elements to control the actual deployment operations (\textit{e.g.}, language elements can be selected and put into practice easily, actions are undoable, error messages that explain recovery methods are available for controlling the deployment operations).}}
	\answer{\Qrating{5}}
	\colplacechunks
	
	\question{\Qitem{\amelia{} has a concise syntax that allows expressing deployment operations in short specification files.}}
	\answer{\Qrating{5}}
	\colplacechunks
\end{parcolumns}

\paragraph{Reliability}
\begin{parcolumns}[colwidths={1=0.76\textwidth},nofirstindent]{2}
	\question{\Qitem{\amelia{} prevents making errors in deployment specifications. The language constructs helps the user to avoid mistakes.}}
	\answer{\Qrating{5}}
	\colplacechunks
	
	\question{\Qitem{\amelia{} includes the right elements and correct relationships between them (it prevents unexpected interactions between its elements).}}
	\answer{\Qrating{5}}
	\colplacechunks
\end{parcolumns}

\paragraph{Maintainability}
\begin{parcolumns}[colwidths={1=0.76\textwidth},nofirstindent]{2}
	\question{\Qitem{\amelia{} is composed of discrete components such that a change to one component has minimal impact on other components.}}
	\answer{\Qrating{5}}
	\colplacechunks
\end{parcolumns}

\paragraph{Productivity}
\begin{parcolumns}[colwidths={1=0.76\textwidth},nofirstindent]{2}
	\question{\Qitem{The development time of writing software deployment specifications is improved.}}
	\answer{\Qrating{5}}
	\colplacechunks
	
	\question{\Qitem{\amelia{} helps to improve the productivity of system deployment.}}
	\answer{\Qrating{5}}
	\colplacechunks
\end{parcolumns}

\paragraph{Expressiveness}
\begin{parcolumns}[colwidths={1=0.76\textwidth},nofirstindent]{2}
	\question{\Qitem{A deployment strategy can be mapped into a \amelia{} specification easily.}}
	\answer{\Qrating{5}}
	\colplacechunks
	
	\question{\Qitem{\amelia{} provides one and only one good way to express every concept of interest.}}
	\answer{\Qrating{5}}
	\colplacechunks
	
	\question{\Qitem{Each \amelia{} construct is used to represent exactly one distinct concept in the application domain.}}
	\answer{\Qrating{5}}
	\colplacechunks
	
	\question{\Qitem{The language constructs correspond to significant application domain concepts. \amelia{} does not include domain concepts that are not important.}}
	\answer{\Qrating{5}}
	\colplacechunks
	
	\question{\Qitem{\amelia{} does not contain conflicting or ambiguous elements.}}
	\answer{\Qrating{5}}
	\colplacechunks
	
	\question{\Qitem{\amelia{} is at the right abstraction level for writing deployment specifications, such that it is not more complex or more detailed than necessary.}}
	\answer{\Qrating{5}}
	\colplacechunks
\end{parcolumns}

\paragraph{Integrability}
\begin{parcolumns}[colwidths={1=0.76\textwidth},nofirstindent]{2}
	\question{\Qitem{\amelia{} can be integrated with other languages used in the software development process, such as using already developed libraries. (\textit{e.g.}, language integrability with other languages).}}
	\answer{\Qrating{5}}
	\colplacechunks
	
\end{parcolumns}
	
\end{document}
